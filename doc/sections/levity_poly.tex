\section{Levity Polymorphism}

Levity Polymorphism, PLDI'17 \citep{eisenberg2017levity}.

\subsection{Motivation}

\begin{verbatim}
bTwice :: forall a. Bool -> a -> (a -> a) -> a
bTwice b x f = case b of True -> f (f x)
                         False -> x
\end{verbatim}

Polymorphic function is supposed to work for any type of argument \verb|x|.
However, the type of \verb|x| influences the calling convention, and hence the
executable code. For example, for list \verb|x|, the function would be passed in
a register pointing into the heap; for a float \verb|x|, it would be passed in a
special floating-point register. 

One simple but very slow solution: represent every value uniformly, as a
pointer to a heap-allocated object.

Most languages also support \textit{unboxed values} that are represented by the
value itself rather than a pointer. Haskell classifies types by kinds. Lifted
types have kind \verb|Type|, while unlifted types have kind \verb|#|.

Current Instantiation Principle: all polymorphic type variables have kind \verb|Type|.
However it introduces several problems. For example,
\begin{verbatim}
-> :: Type -> Type -> Type
\end{verbatim}
means \verb|Int# -> Int# -> Int#| is ill-typed.
The current design is sub-kinding:

\begin{verbatim}
Type <: OpenKind
#    <: OpenKind
->   :: OpenKind -> OpenKind -> Type
\end{verbatim}
which is awkward and unprincipled.

The idea of this paper is to \textit{replace sub-kinding with kind
  polymorphism}. The main idea is

\begin{verbatim}
-- primitive
TYPE :: Rep -> Type   

-- definitions
type Rep      = [UnaryRep] -- a type-level list
data UnaryRep = PtrRep     -- boxed, lifted
              | UPtrRep    -- boxed, unlifed
              | IntRep     -- unboxed ints
              | FloatRep   -- unboxed floats
              | DoubleRep  -- unboxed doubles
              | ...etc...
type Lifted   = '[PtrRep]
type Type     = TYPE Lifted
\end{verbatim}

\subsection{Notes}

 \begin{tabular}{p{4cm}|p{4cm}|p{4cm}}
   \hline\\
   & \textbf{boxed}: represented by a pointer into the heap & \textbf{unboxed}: represented by the value itself \\
   \hline\\
   \textbf{lifted}: lazy; has one extra element beyond the usual ones representing a non-terminating computation (call by need) & \verb|Int|, \verb|Bool| & (Haskell represents lazy computation as thunks, so lifted can only be boxed) \\
   \hline\\
   \textbf{unlifted}: strict (call by value) & \verb|ByteArray#| & \verb|Int#|, \verb|Char#|\\
   \hline
 \end{tabular}
  
\begin{itemize}
\item Any type that classifies values has kind \verb|TYPE r| for some
  \verb|r :: Rep|. The type \verb|Rep| specifies how a value of that type is
  represented, by  giving a list of \verb|UnaryRep|. A \verb|UnaryRep| specifies
  how a single values is represented. Examples:
\begin{verbatim}
Int    :: TYPE '[PrtRep], TYPE Lifted, Type
Int#   :: TYPE '[IntRep]
Float# :: TYPE '[FloatRep]
(Int, Bool)  :: Type
Maybe Int    :: Type
Maybe        :: Type -> Type
\end{verbatim}
\item Levity polymorphism: an abstraction over only the levity (lifted vs.
  unlifted) of a type.
\begin{verbatim}
(->) :: forall (r1 :: Rep) (r2 :: Rep).
        TYPE r1 -> TYPE r2 -> Type
\end{verbatim}
\item Restrict the use of levity polymorphism so that it can be compiled:
  \begin{itemize}
  \item Disallow levity-polymorphic binders.
  \item Disallow levity-polymorphic function arguments.
  \end{itemize}
\item The correctness of levity polymorphism is proved by 1) defining $\mathcal{L}$:
  a variant of System F that supports levity polymorphism 2) defining
  $\mathcal{M}$: a $\lambda$-calculus in A-normal form, with operational
  semantics working with an explicit stack and heap; 3) a type-erasing
  compilation from $\mathcal{L}$ to $\mathcal{M}$, with correctness proofs.
\end{itemize}

%%% Local Variables:
%%% mode: latex
%%% TeX-master: "../doc"
%%% org-ref-default-bibliography: "../doc.bib"
%%% End: