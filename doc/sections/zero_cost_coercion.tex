\section{Safe Zero-cost Coercions for Newtypes}

Safe Zero-cost Coercions for Haskell, ICFP'14
\citep{Breitner:2014:SZC:2628136.2628141}, JFP'16 \citep{breitner2016safe}.

This work itself is built on Generative Type Abstraction and Type-level
Computation, POPL'11 \citep{Weirich:2011:GTA:1926385.1926411}.

\subsection{Motivation}

Consider to define a newtype \verb|HTML|

\begin{verbatim}
newtype HTML = MkHTML String

unHTML :: HTML -> String
unHTML (MkHTML s) = s

linesH :: HTML -> [HTML]
linesH h = map MkHTML (lines (unHTML h))
\end{verbatim}

The runtime cost of \verb|linesH| is inevitable. To avoid that, this paper
describes safe coercions, with the constraint \verb|Coercible| and the function

\begin{verbatim}
coerce :: Coercible a b => a -> b
\end{verbatim}

Then \verb|String -> [String]| and \verb|HTML -> [HTML]| would be coercible and
the function can be rewritted to

\begin{verbatim}
linesH :: HTML -> [HTML]
linesH = coerce lines
\end{verbatim}

\verb|Coercible| is translated into \FC, augmented with \textit{roles}.

\subsection{Notes}

\begin{itemize}
\item \verb|Coercible s t|: \verb|s| and \verb|t| have bit-for-bit identical
  run-time representation.
\item For every type constructor, each type parameter has a role, determined by
  the way in which the parameter is used in the definition of the type constructor.
  \begin{itemize}
  \item representational: type parameters of ordinary newtypes and datatypes
  \item phantom: it does not occur in the definition of the type, or it occurs
    only as a phantom parameter of another type constructor.
  \item nominal: parameters that possibly affect the run-time representation of
    a type, including parameters of a type/data family, non-uniform parameters
    to GADTs, type classes.\\
    Parameters of type variables are always nominal.
  \item Users can specify the roles via annotation, and the compiler ensures that
    role annotations cannot violate type safety.
    \begin{verbatim}
type role Map nominal representational
    \end{verbatim}
  \end{itemize}
\item To decide whether two types are coercible:
  \begin{itemize}
  \item \textit{unwrapping} rule: for every \verb|newtype NT = MkNT t|,\\
    we have \verb|Coercible t NT|.\\
    This rule is available only if the corresponding newtype data constructor is
    in scope.
  \item \textit{lifting} rule: for every type constructor \verb|TC r p n|,\\
    if \verb|Coercible r1 r2|,\\
    we have \verb|Coercible (TC r1 p1 n) (TC r2 p2 n)|.
  \item Coercible is an equivalence relation: reflexivity, symmetry, transitivity.
  \item \textit{decomposition} rule: given non-newtype \verb|T|,\\
    if \verb|Coercible (T r1 p1 n1) (T r2 p2 n2)|,\\
    then \verb|Coercible r1 r2|, and \Verb[commandchars=\\\{\}]|n1 \mytilde n2|.
  \item \textit{type application} rule: If \verb|Coercible t1 t2|,
    where \verb|t1, t2 :: k1 -> k2|,
    then \verb|Coercible (t1 x) (t2 x)|.
  \end{itemize}
\item The type system in the paper is more like the one in Section~\ref{sec:fcpro}:
  types and kinds are not unified and coercions are homogeneous.
\item New form of coercion: $[[GG |-co co : s1 ~ (k, p) s2]]$
  \begin{itemize}
  \item Nominal equality, written $[[~N]]$.
    \begin{itemize}
    \item The equality that the source Haskell type checker reasons about.
    \item Type families introduce new nominal equalities.
    \end{itemize}
  \item Representational equality, written $[[~R]]$.
    \begin{itemize}
    \item The equality holds between two types that share the same run-time representation.
    \item A subset of nominal equality.
    \item A \verb|Coercible| constraint corresponds to a proposition of
      representational equality.
    \item New types introduce new representational roles.
    \end{itemize}
  \item Phantom equality, written $[[~P]]$, holds between any two types.
  \end{itemize}
\item Important type-safe cast:
  \[ \drule{fc-typing-Cast-R} \]
  The coercion $[[co]]$ must be a proof of representational equalities.
\end{itemize}
%%% Local Variables:
%%% mode: latex
%%% TeX-master: "../doc"
%%% org-ref-default-bibliography: "../doc.bib"
%%% End: