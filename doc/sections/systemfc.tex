\section{\FC}

System F with Type Equality Coercions, TLDI'07 \citep{Sulzmann:2007:SFT:1190315.1190324}.

\subsection{Motivation}

Language designers have begun to experiment with a variety of type systems that
are difficult or impossible to translate into System F, such as functional
dependencies, generalized algebraic data types (GADTs), and associated types.

This paper presents \FC, which extends System F with 1) explicit equality
witnesses; 2) non-parametric type functions.

\subsection{Notes}

\begin{itemize}
\item The role of coercion in typing:
  \[ \drule{fc-typing-Cast} \]
\item The systems merges types and coercions.
  \begin{itemize}
  \item Types have judgment $[[ GG |-ty s : k ]]$
  \item Coercions have judgment $[[ GG |-co co : s1 ~ s2 ]]$. Homogeneous.
  \end{itemize}
\item Kinds $[[ k ]] ::= [[star]] \mid [[k1 -> k2]] \mid [[s1 ~ s2]] $.
\item Sorts $[[ sort ]] ::= [[ TY ]] \mid [[ CO ]]$.
  \begin{itemize}
  \item  $[[ TY ]]$ for kind $[[ star ]]$ and $[[ k1 -> k2
    ]]$;
  \item  $ [[CO]] $ for $[[ s1 ~ s2 ]]$ and $[[ co1 ~ co2 ]]$.
  \end{itemize}
\item Coercions $[[co]]$ are types, $[[s1 ~ s2]]$ are kinds, $[[CO]]$ are sorts.
  $[[co]] :: [[s1 ~ s2]] :: [[CO]]$.
\item The meaning of type function is given by axioms.
\item Type functions are required to be saturated.
\end{itemize}

%%% Local Variables:
%%% mode: latex
%%% TeX-master: "../doc"
%%% org-ref-default-bibliography: "../doc.bib"
%%% End: